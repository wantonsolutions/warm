\section{Introduction}

Resource disaggregation, and specifically memory disaggregation incurs serious
performance penalties due resources being exclusively accesses over a network. In
the case of memory disaggregation the latency of accessing remote memory over
local is approximately 20x, (50ns local and ~1us remote). While this cost may
seem prohibitive the benefits of resource disaggregation, mainly high memory
utilization, arbitrarily flexible resource allocation, and energy efficiency make
the architectural paradigm tantalizing.

Protocols for accessing remote memory vary (RDMA (ROCE, IB)~\cite{},
GenZ~\cite{}, omipath~\cite{}) however each protocol meets approximately the same
requirements, and services the same needs. Reliable access to by addressable
remote memory with low latency and high throughput.

Regardless of the memory transport technology the underlying scheme is the
same, clients request remote addresses, and the address translation and memory
operation serving is performed by DMA capable NICs (either commodity RDMA capable
NICS ConnectX5/6 or smartNICs with configurable protocols Cavium~/cite{}, Clio
~/cite{clio}). While the implementation of the memory access semantics, when
applied to the context of resource disaggregation the remote memory operations
are handled exclusively by the network interface cards. This is because the
machines hosting the memory are by definition not housing a CPU which could
otherwise be used to aid in the memory access operations~\cite{HERD,SoNUMA,
STORM, (every KV store), RMA-paper}. An implication of running all memory
operations on a NIC is that even while being colocate across a PCIe bus from
the memory it controls, it still incurs a serious latency penalty for
arbitrating across the PCIe bus.

The penalty for controlling remote memory without the aid of a remote CPU to aid
in serialization of memory operation is stark. Clients are essentially forced
strictly partition their accesses to remote memory as the price of contention is
high~\cite{LegoOS, Remote Regions}. Systems which aim to offer shared remote
resources to clients in a disaggregated setting pay the harsh performance
penalty of relying on NICs as memory controllers. Nowhere is this cost clearer
than with the use of atomic memory operations on the NIC. The performance of
RDMA atomics has been measured in prior work ~\cite{requirements for
RDMA}~\cite{Clover}. In state of the art memory disaggregation systems with
sharing such as Clover~\cite{clover} a great deal of care is taken to avoid both
memory conflicts and the need to perform multiple reads or writes to complete an
operation. To achieve this Clover caches metadata about the location of the
latest writes and reads while also make use of a remote data structures which
allows for lockless reads. In the case of highly contended resources however the
performance of clover diminishes sharply due to an increased number of atomic
locking operations required on reads.

In this work we make the observation that given a rack-scale disaggregated
system memory operations are serialized by a centralized interconnect (either a
Switch or some other middlebox~\cite{disaggregation and the application}). This
physical layer serialization imposes a globally observable total order on memory
operations within a rack. As such this centralized location can be used to
resolve contention to shared resources, provide locking, and remove the need for
expensive atomic operations on an RDMA enabled NIC. To show the benefits of such
a system we implement a variety of in network contention resolutions for the
Clover protocol and demonstrate their performance boots. Specially we cache the
locations of the most recent reads and writes with stale metadata are steered to
the most up to date locations allowing for O(1) remote memory accesses in all
cases.

We demonstrate in network serialization removes the need for expensive locking
operations on the NIC. Using RDMA transport information and clover specific
application knowledge all reads and writes to contended areas are totally
ordered in the network. Specifically all reads and writes to the same keys are
multiplexed to the same queue pairs, by utilizing the RDMA ordering requirements
of QP's reads and writes require no expensive locks and can flow at line rate to
remote memory. This ordering requires a number in band adjustments to the RDMA
protocol in order to interoperable with commodity hardware. QP state must be
maintained in network, specifically the sequence numbers of multiplexed
requests, so that response packets can be demultiplexes back to their original
connections. Small adjustments such as generating acks for collapsed requests is
also required. We demonstrate that these algorithms are implementable in network
at little cost with a DPDK prototype. We measure that ~\todo{we achieve a ?X
improvement in performance using only XMB of in network state, and ?X
performance improvement in highly contested settings with full use of system
memory}.