\section{Background}


\subsection{Resource Disaggergation}

Resource disaggregation is an archetectural paradigm which separates resources
such as disk, cpu and memory over a network ~\cite{requirements for resource
dissagregation}~\cite{legoOS}. The aim of resource disaggregation is to allow
for near limitless flexibility in terms of machine composition. Memory can be
dynamically added and removed from a system by reconfiguration, rather than by
manually changing the physical components of a single machine~\cite{fastswap}.
It is now common case for disks (HHD and SSD) to be disagregated from CPU and
memory by a network ~\cite{decible}. Disks are 

\subsection{RDMA Key Value Stores}

\subsection{Programable middleboxes}

\subsection{Clover}
