\section{Introduction}

There has been tremendous interest in resource disaggregation in
recent years, with both academic and industrial researchers chasing
the potential for increased scalability, power efficiency, and cost
savings~\cite{blade-server,fastswap,rethinking,the-machine,requirements,clio-arxiv,firebox,leap,zombieland,storm,aifm,legoos,supernic}.
By physically separating compute from storage across a network, it is
possible to dynamically adjust hardware resource allocations to suit
changing workloads.  Considerable headway has been made at higher
levels of the storage hierarchy; published and even production systems
support remoting spinning disks, SSDs, and modern non-volatile
memory technologies~\cite{decible}.  Remote primary storage---a.k.a.
memory pooling---remains a fundamental challenge, however, due to the
orders-of-magnitude disparity between main-board access latency and
even intra-rack round trips.

%Resource disaggregation is an architectural paradigm which separates
%disk, CPU and memory over a network. The goal of this architecture is
%to enable extreme flexibility in terms of machine composition.  For
%example a systems memory capacity can be dynamically apportioned by
%reconfiguration, rather than by manually changing the physical
%components of a single machine. It is now common for disks (HHD and
%SSD) to be disaggregated from CPU and memory. SSDs are comparatively
%easier to disaggregated than main memory as their access latencies are
%on the order of 10's of microseconds which amortizes the network round
%trip cost.

%Local memory latency is around 50ns. The cost of accessing memory over the
%network is on the order of 1us -- approximately a 20x overhead. This order of
%magnitude difference in latency makes hiding remote memory accesses a hard
%problem.  

The hardware community has made great strides in closing the latency
gap via novel technologies like silicon photonics and new rack-scale
interconnects, but commercially available options remain significantly
slower than on-board alternatives.  Concretely, while industrial consortia
have proposed cache-coherent memory technologies~\cite{genz,cxl} that
would dramatically lower access latencies, currently available
interconnects based on RDMA~\cite{infiniband-spec}
%%
%\todo{distinguish CXL 200-500ns latencies have been proposed}
%%
remain on the order of 20$\times$ slower than a local access (e.g.,
50~ns local versus 1~$\mu$s remote).  As a result, despite the fact
that current-generation memory transport technologies provide the
ability to directly execute requests like read, write, and compare-and
swap-on remote host memory through the use of RDMA-capable
NICs~\cite{connectx}, SoCs~\cite{cavium}, FPGA
SmartNICs~\cite{corundum,kv-direct}, or DPUs~\cite{fungible}, most
existing systems coordinate with a remote CPU on the socket at which
the DMA is being performed to assist with
serialization~\cite{cliquemap,erpc,herd,sonuma,storm}.
%%


%% % and Omni-Path~\cite{omni-path}. // omni-path is dead now % 
%Each protocol, while distinct, meets approximately the same requirements,
%reliable access to byte addressable remote memory with low latency and high
%throughput.

%% it would be nice to Cite SUPERNIC and CLIO here but I'm not sure it makes
%sense untill it's published at a major venue
%%Clio~\cite{clio-arxiv}
%%todo ask alex about the archive reference
%%todo do a quick read of how DMA is dealt with on the other interconnects

In the absence of a general-purpose CPU located alongside remote
memory, it falls to each individual client to ensure that its reads
and writes are serialized, usually by leveraging expensive
hardware-provided atomic operations at the server like
compare-and-swap (CAS)~\cite{design-guidelines} as the latencies
involved in client-side coordination are prohibitive.  As a result,
most existing systems simply partition memory completely and forgo
sharing~\cite{reigons,fastswap,legoos}.
%%
%The few published systems that provide fully passive remote memory
%target scenarios involving read-heavy workloads~\cite{clover}, client
%colocation~\cite{sherman}, or memory-inefficient
%datastructures~\cite{race} \textbf{XXX:Need to say more} where the
%costs of conflict detection and resolution can be effectively
%amortized.
%
The few published systems that do support shared access mediate
requests to specialized data structures~\cite{clover,sherman}.

For example, Sherman, a write-optimized
B+Tree~\cite{sherman} places its locks in NIC memory at the server to
avoid crossing the remote PCIe bus.
%, resulting in 3$\times$ higher throughput.
Clover~\cite{clover} implements a hash table that
supports lock-less reads; concurrent writes are supported through a
client-driven optimistic concurrency protocol.  Despite their clever
designs, however, both approaches simply delay the inevitable:
%%
Sherman's NIC-based locks are subject to significant hardware limits
imposed by the CAS operation required to enforce serialization.
Similarly, Clover's client-based recovery scheme quickly becomes cost
prohibitive when faced with non-trivial levels of write contention.
%
%it is repeatedly executed on a single
%address (Lock acquire and release). Further, Sherman requires that cliques of
%clients are colocated to resolve most contention based conflicts locally.
%%
% CAS instructions are only executed to commit
%writes and never land on the same address twice - effectively bypassing the
%single address limits of CAS.
%%
%While highly scalable for read-heavy workloads,  This

We argue that these shortcomings are not unique to the particular
systems, but rather fundamental to any approach that implements
distributed conflict resolution.
%
%% minimize conflicts by caching  metadata about the
%% location of the latest writes and reads while also make use of a
%% remote data structures which allows for lock-less reads. In the case of
%% highly contended resources however the performance of clover
%% diminishes sharply due to an increased number of atomic locking
%% operations required on writes.
%
The obvious alternative is to deploy a centralized memory controller
(e.g., a CXL 2.0 switch) that can serve as a serialization point and
ensure all races are resolved before accessing memory, but effective
realization of such a design has proven elusive.  While many proposals
exist, none of them have yet been implemented in commercially
available hardware.  More to the point, such designs are inherently
unscalable as they require all accesses to be managed by the
controller, rather than forwarded directly between the client and
relevant server.

In this work we make the observation that such a serialization point
already exists in today's rack-scale disaggregated deployments: the
top-of-rack switch.  We propose to leverage the capabilities of modern
programmable switches to cache sufficient information about in-flight
requests to transparently detect and resolve conflicts before they
occur.  Unlike a centralized memory controller, however,
our serializer does not need to operate on---or even maintain state for---all
remote memory requests.  First, it only needs to address actual conflicts
and can avoid the unnecessary costs of enforcing ordering among
unrelated requests.  Second, in deployment scenarios where it may
lack the resources to track all requests, our serializer can serve as a
performance-enhancing proxy: when deployed alongside client-based
conflict resolution techniques, it can allow even conflicting requests
to pass through unmodified without jeopardizing safety while
decreasing the frequency of conflict resolution.

We present {\sword}, an on-path serializer
%(implemented either
%directly on the top-of-rack switch or an attached
%middle box~\cite{disandapp})
that dramatically improves the performance of remote memory systems
that support write sharing.  Like all ToRs, {\sword} imposes a
globally observable total order on memory requests (i.e., packets)
within a rack.  In scenarios where {\sword} has the resources to
explicitly manage RDMA connections, it can enforce per-server ordering
at the ToR and remove the expensive CAS operations from all in-flight
packets, avoiding their associated performance bottlenecks entirely.
More generally, however, {\sword} can be deployed alongside an
underlying optimistic concurrency scheme: remote memory operations
remain guarded to ensure that clients can detect and recover from
conflicts of which {\sword} may not be aware.  Instead, because
{\sword} understands the disaggregated memory protocol, it can keep a
cache of recent operations to adjust subsequent requests whose
guards it knows are doomed to fail.  In such cases,
%If suitably provisioned (i.e., it has the
%appropriate metadata cached),
it modifies requests in
flight to account for the preceding operations and decrease the
likelihood the guard will trip.

We prototype {\sword} in two scenarios using a rack of servers
equipped with ConnectX-5 RoCE-enabled NICs.  First, we use a
DPDK-based implementation to replace CAS requests in flight with
standard RDMA verbs, allowing systems like Sherman to overcome the
hardware limit on atomic requests per queue pair.  Second, we
implement a lightweight version of \sword\ on a P4 programmable switch
to accelerate Clover's optimistic concurrency protocol. Our evaluation
shows that {\sword} dramatically increases the performance of Clover
in the presence of write contention: Under a 50:50 read-write
workload, throughput rises by almost 35$\times$ while bandwidth usage and tail latency drop by 16 and 300$\times$.

%% Using RDMA transport information and clover specific application
%% knowledge all reads and writes to contended areas are totally ordered
%% in the network. Specifically all reads and writes to the same keys are
%% multiplexed to the same queue pairs, by utilizing the RDMA ordering
%% requirements of QP's reads and writes require no expensive locks and
%% can flow at line rate to remote memory. This ordering requires a
%% number in band adjustments to the RDMA protocol in order to
%% interoperable with commodity hardware. QP state must be maintained in
%% network, specifically the sequence numbers of multiplexed requests, so
%% that response packets can be demultiplexes back to their original
%% connections. Small adjustments such as generating ACKs for collapsed
%% requests is also required. We demonstrate that these algorithms are
%% implementable in network at little cost with a DPDK prototype. We
%% measure that ~\todo{we achieve a ?X improvement in performance using
%%   only XMB of in network state, and ?X performance improvement in
%%   highly contested settings with full use of system memory}.
