\section{background}

\subsection{Disaggregated Memory}

Disaggregated memory is primary storage separated from a CPU
by a fast network. The composition of disaggregated memory
systems are up for debate -- transparency vs explicit remote
access divides the field. Transparent approaches typically
utilize the virtual memory system, using a remote memory
bank to service page faults~\cite{infiniswap, leap,
fastswap}. Some have argued that page granularity is too
course and built upon an incoherent caching
layer~\cite{kona, legoos}. In both cases transparency costs
performance as remote accesses cannot be optimized for by
the running application. Explicit remote access functions
similar to RPC's with RDMA verbs acting as a substrate for a
higher level api~\cite{aifm,reigons,clover,sherman,faast}.
Direct explicit access allows for higher application level
performance as remote requests can be batched, scheduled,
and accounted for by libraries and runtimes. In both cases
sharing is expensive, at the time of writing no transparent
system supports shared memory with writing as the cost of
cache coherence is prohibitively high in terms of bandwidth
and latency. Explicit cases are more promising for sharing
as they allow developers to explicitly acquire locks, and for
libraries to make use of optimistically concurrent
datastructures.

\subsection{RDMA protocol}


%% RDMA
Remote direct memory access (RDMA) allows hosts to directly
access each others memory. The RDMA protocol is a set of
verbs which abstract remote memory accesses. The underlying
protocol is entirely managed by the NIC which exposes the
verbs API to the CPU. By managing the protocol the NIC saves
precious CPU cycles and allows for 100Gbps+ networking while
requiring little to no CPU overhead. The CPU registers
memory regions for DMA with the NIC and sets up connections
(Queue Pairs) with remote RDMA enabled NICs. RDMA
connections come in a variety of flavors, each of which
enables a different set of RDMA verbs. Unreliable Datagrams
(UD), and Unreliable Connections (UC) (UD) for example
operate similar to UDP with no reliable ordering guarantees.
Reliable Connected (RC) operates similar to TCP, the nic
manages connection states for each QP and ensures reliable
delivery by using sequence numbers and a go-back-n
retransmission protocol Reliable Connected (RC) operates
similar to TCP, the nic manages connection states for each
QP and ensures reliable delivery by using sequence numbers
and a go-back-n retransmission protocol.  Serious debate
exists over which connections to
use~\cite{storm,cell,herd,faast,farm}, each has advantages
and disadvantages in terms of NIC resource utilization,
throughput, and latency. For the purposes of disaggregation,
RC connections are the most attractive as unlike UD, and UC,
they allow \textit{Read}, \textit{Write}, and \textit{CAS}
operations to be issued to a remote machine without any
remote CPU acting as an arbiter. From the perspective of the
application developer, remote memory acts similar to local
memory, but with a restricted RDMA interface, rather than a
wide CPU ISA. RDMA atomic instructions are used to implement
serialization without the need for a remote CPU, when an
atomic instruction is executed the memory side NIC examines
operations on all QPs and ensures that the atomic operation
executes transitionally. The cost of executing atomics is
high as it requires QP buffering and a PCIe round trip
before the NIC can respond to the request. Atomic requests
are much slower than Read and Write instructions because of
these two factors. Mellanox NICs supply a small (few KB)
region of device mapped memory which allows the NIC to
execute RDMA instructions on device memory without the need
for a PCIe round trip. Executing atomics on this memory is
~\todo{5x} higher throughput than host memory, but still
slower than it's read and write counterparts. Some projects
have made use of this memory region to accelerate remote
memory locks~\cite{sherman}


%% RDMA Connections
%% RDMA Atomics


\subsection{Programmable Switches}
Many disaggregated memory proposals are rack-scale. They
propose a single rack with servers of CPUs, banks of memory,
and SSD's, each of which is interconnected by a TOR. This
central role of the switch has not gone unnoticed and
disaggregated architects have proposed using modern
programmable switches to enable remote memory apis, and OS
functionality~\cite{disandapp,mind}. Literature on
offloading OS, and service level functionality to
programmable switches is plentiful Literature on offloading
OS, and service level functionality to programmable switches
is plentiful~\cite{netlock,netkv,netchain,netcache}, the
constraints in each case are similar, switches have limited
memory, and processing capabilities, if the computational
ask of the switch is too high packets must be recirculated
adding additional latency, and reducing aggregate bandwidth.
Ideal applications for programmable switches use little
memory, require little processing and enable a huge
performance benefit from centralization, and the billions of
operations (in terms of packets) that a switch can process
per second.

Prior work has shown that programmable switches are able to
manage locks~\cite{netlock}, track the state required to
maintain an RDMA reliable connection~\cite{tea}, and provide
rack scale serialization at low
cost~\cite{eris,no,when-computer}. These properties make a
top-of-rack programable switch ideal for managing remote
memory as it can guard access, maintain connections, and
provide serialization primitives for all clients.


\section{Serialization}

\subsection{RDMA serialization}
RDMA provides a set of atomic instructions for serializing
access to remote memory without the need for a memory side
CPU. These instructions \textit{Fetch-and-Add} and
\textit{Compare-and-Swap} (CAS) execute on 64 bit width
words and are guaranteed to be consistent across queue pairs
with atomic visibility to all observers. These operations
are expensive. On a single address atomic instructions can
only execute a few million operations per
second~\todo{Figure 1}, across independent addresses they
have approximately half the throughput of their non atomic
counterparts. These limitations fundamentally limit data
structures built with them, locks with a single address
requiring both a lock and unlock call to guard access are
limited to around 500k accesses per second. This assumes
perfectly coordinated access, under contention even failed
requests, attempting to aquire a taken lock with CAS, eats
into the operation bandwidth. Under contention RDMA has poor
support for traditional locking. In contrast optimistic data
structures with locks scattered throughout, such as a linked
list, are not rated limited by this single address
restriction. However, they are fundementally limited by the
fact that any atomics have half the throughput of reads and
writes. More critically, under contention optimistic data
structures have no liveness guarantees under contention
atomic operations fail frequently~\todo{Figure: failure to
succeed}. RDMA has no support for resolving pointers
(pointer chaising) and retrying operations for such data
strucutres(~\cite{rma}), a simple operation usually executed
by a memory side CPU. As such RDMA clients are forced to
resolve their failures requests themselves often retrying
many times.

Atomic operations are not the only serialization mechanism
provided by RDMA. Reliable connections provide in order
delivery on individual queue pairs. This allows clients to
issue multiple requests in parallel and allows the NIC to
resolve reordering and dropped packets using go-back-n This
allows clients to issue multiple requests in parallel and
allows the NIC to resolve reordering and dropped packets
using go-back-n retransmission. When clients are collocated,
queue pairs can be shared by multiple cores through
techniques like flat combining~\cite{flock,sherman}. This
technique removes the need for RDMA atomics, as clients can
locally resolve their conflicts and then issue their
requests to remote memory at the full throughput of reads
and writes. Unfortunately this technique is not applicable
to distributed clients as they do not share access to the
same queue pair

\subsection{Programmable Switch Serialization}

Switches can cheaply serialize packets~\cite{when-computer}.
Programable switches with P4 pipelines can use this cheap
serialization, along with their ability to manage small
amounts of in network state, to serialize distributed
applications. Mind for instance provides a unified TLB and
cache for disaggregated applications~\cite{mind}. Packets
processed by a programmable switch are sequenced in order,
updates to the switches registers are atomic with respect to
the packets as each state of the pipeline is occupied by
exactly one packet at a time. A centralized switch can
therefore apply monotonic sequence numbers to a stream of
packets, or maintain a lock without the need for explicit
atomic operations.

Unfortunately this serialization is not sufficient for RDMA
memory operations out of the box. RDMA packets on two
reliable connections may be ordered on the switch, and then
subsequently reordered by the receiving side NIC, or by the
PCIe bus~\cite{understanding-pcie}.

In all cases switch memory and compute are highly
constrained. Any operations that exceed the processing
limit of the match action pipeline require recirculation
which consumes additional switch bandwidth. For example,
terminating connections with a switch is prohibitively
expensive as both the state of the entire connections, and
the logic for connection startup and teardown would consume
large quantities of the switches buffers. Any logic, or data
offloaded to a programable switch must therefore be minimal
in order to meet the compute and data restrictions of the
switches SRAM.

\section{Swordbox}

We present \sword a middlebox solution for sharing remote
memory. \sword provides acceleration for both lock based and
optimistically concurrent data structures. Locks are managed
in network, and the results of locking operations are
forward to memory. Lock accesses are accelerated by
replacing RDMA atomic operations with writes in flight.
Modified lock operations are multiplexed onto existing RDMA
connections, per lock, to ensure serialized access to the
lock. \sword also provides a mechanism to remove contention
from optimistic data structures by caching the metadata
required to detect and resolve conflicts. These techniques
are implemented in a DPDK prototype for the lock based
approach and fully realized in P4 for optimistic data
structures.

\subsection{Caching}

\subsection{Locks}

\subsection{Optimistic Concurrency}