\begin{abstract}

Effectively sharing passive remote memory remains an open problem.
% While emerging standards like CXL promise cache-coherent memory
% pooling, spec-compliant hardware is not yet commercially
% available and its feasibility at scale remains unproven.
Despite its promise, experience with commercial RDMA hardware has
shown that 1-sided verbs are incapable of managing concurrent updates
at scale.  Faced with this reality, most prior systems either disallow
sharing, retreat to 2-sided verbs---which require co-locating computational
resources with remote memory---or propose new RDMA
hardware.  We observe that there is an alternative
approach.
   
% Existing RDMA-based systems employ optimistic concurrency
% approaches that defer serialization to the remote memory server and
% rely upon heavyweight RDMA atomics to ensure consistency.
% Unfortunately, atomic operations scale poorly causing these approaches
% to degrade rapidly under contention.

We present \sword, an approach to leveraging the \textit{de facto}
serialization point in rack-scale disaggregated systems---the
top-of-rack switch---to transparently resolve RDMA races in flight.
\sword\ runs on programmable switch hardware and accelerates existing
RDMA-based approaches in two ways.  In cases where it can interpose upon
all requests to a given server, \sword\ multiplexes multiple clients'
RDMA requests onto shared reliable connections to ensure operation order and remove the need for heavyweight
atomic operations.
%and overcome their hardware performance bottlenecks.
More generally, it can track access patterns and dynamically steer
RDMA-based updates to append-only data structures commonly employed in
shared-write situations to avoid triggering expensive client-based
resolution.  We apply our P4-based prototype to Clover, a
state-of-the-art RDMA-based disaggregated key-value store.  Under a
YCSB-A workload, throughput rises by nearly $35\times$ while bandwidth
usage and tail latency drop by 16 and 300$\times$, respectively.

\end{abstract}
