\section{Scratch}
\label{sec:scratch}

Using memory as a syncronization tool is a decent tactic as the writes
to a location are sequential, however the end hosts can only be
rejected. Even if we innovate on this technololy some performace is
dropped on the floor as an entire round trip must be made for the
rejection notification to be relayed and the notification messages to
be generated.

A centralized switch solution is fully capable of implementing each of the in
memory operations while providing serialization. The key insight is that
memory operations can be serialized at the switch centrally. The last moment
for reordering is egress on the switch port to an endhost (assuming that the
nic preserves ordering, queue pairs are maintained, and the memory controller
does not mess with us). 

todo{Follow the path of a packet and determine where reordering could
happen at each point}

lowest level, switches provide serialization, versioning of memory, and notify multicast, and write rejection.
Programmable switches can provide serialization at the egress port, they can
maintian versioning of memory by kepping an atomic counter on each block.
Switches can mantain notify groups for each block. Using multicast primitives
notify messages can be propagates to cores sharing memory as fast as a direct
message. 

locking and transactional memory

Locking is as simple sending a lock request to a block, the switch either
propagates lock messages or rejects the lock if the version of the write is
out of date.

The switch can support transactions by collecting all memory writes and reads
and write of a packet. A single reject fails the transaction, while a
collective accept into a single packet header with itterative packet
processing for each read This is equivalent to the atomic collection of
multiple locks.

%% Race condtion advantage Without a syncronization mechanism si clients must
determine themselves if their wreads a vaild. Proposed techniques include
using a checksum of the memory values as validation that the read or write
occured to the same value of memory that was observed prior, another
technique is to use hashing to validate that the version of the memory
witnesses is identical to the expected value. This is in the case where no
lock on the memory is acquired. Aquiring a lock on a memory location requires
that a request is made to the lock location, the read or write is performed
and then the lock is released. In the case of RDMA one sided operations this
can take up takes 3 RTT in the normal case, and more twhen the lock is
contested. In general without a centralized athority starvation can occur.
This problem is only made more severe when multiple memory locations need to
be accessed as each location requires a lock to be accessed and then
released. This can lead to priority inversion, lock starvation, and live
lock. 

A memory centric archetecture utilizes the fact that a single memory location
serilaizes its own access. Versions of memory can therefore be controlled
using a simple counter, where each write increments the version of the
memory. Client request to the memory can test and set the version number to
ensure consistancy. However this requries that memory is able to perform such
a check. Using memory as a syncronziation mechanism aleviates clients of the
need to check the consistancy of their reads and writes. Clients need only
submit the version of the memory they wish to read or write, and the memory
can determine if the read or write is valid. From a clients perspective it
takes a single RTT for a memory access. This does not however solve the
problem of transactional batching of reads and writes. A client still needs
to lock the locations, individually, then perform an atomic operation, and
then unlock. This reuqires an RTT for each read and write (Storm 5.4)
~\cite{storm} A programmable switch can currently perform each of the
operations a memory centric archetecture can only promise after new
technologies arrive, and it can perform it with lower latency. While a client
requires a full RTT to to determine if its reads and writes are vaild, and a
full RTT to be notified if a memory location has changed, a switch, at the
top of a rack, can return invalid reads and writes at half and RTT, and
notifiy other clients that memory has been modified in half an RTT.

Notify latency - Memory centric architectures propose that memory should
notify cores which share a region of memory when the memory is updated. This
is to prevent them failing to make a vaild read. The delay for such a
notification is a single RTT as the memory must first accept a new write and
then broadcast to all those affected nodes. A programmable switch, can
disseminate such a notification in half the time leading to far less
contention and out of date reads and writes.

Notify Groups - Keepying track of which cores should be notified is a probelm
on its own. The HP crew suggest a publish subscribe model. This would require
each piece of memory to have a group associated with it. Given that the
memory is going to be calling the group based on a notfiy, the memory must
track a group of nodes for each memory region. This problem is true on the
switch as well, and perhaps worse as the switch is more resource constrained.
I think that groups should be difficult to create, i.e a consensus protocol
should be invoked before a node can read or write to a reigion of memory.
Then it is the job of the nodes to know who should be notified for a given
read or write. The IPs of the group could then be put into a multicast group
so that the switch need only forward the request to the subscribed nodes when
a write occurs. Consensus should also be used to initally spread state about
the shared region and RDMA queue pair information.

Serialization probelms - The key insight of this work is that the egress port
on the switch can at as a serialization point just as well as memory.
However, this insight hinges on the idea that no other reordering can occur.
Is this an okay assumption to make? It requires that there is no reordering
at the NIC, and that the RDMA requests from the NIC to memory are also not
reorderd. The path is Switch Port - > NIC -> DMA engine (potentially with
scatter gather) -> Local memory controll (who knows what these things
actually do). To make this happen a deep understanding of each of these
components and the potential for requests to run in parallel must be known.


Granularity - It is impracticle to assume that we would ever have byte level
access to remote memory. The overhead is too high. Judicious though should be
put into how large the block size for remote memory should be. The smallest I
can imagine it would be a cache line. Ideally it would be the size of a page.
\todo{Find a citation about pageing to remote memory (GFS / legos)}

\section{Use Cases}

Any in memory RDMA store. The B-Tree os one example, memcached D is another.
OS structures are another potential. It would be cool to have a file buffer
cache with both metadata (local) and file data (remote) as a use case. The
key thing here is that I want to both enable distributed applications, and
the development of distributed OS's. This sould be seen as an interface to
shared memory and not a specific shared memory sysetm (Although it will be).
