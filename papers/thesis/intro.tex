% \section{A Case for \sword}

% There has been tremendous interest in resource disaggregation in recent years, with both academic
% and industrial researchers chasing the potential for increased scalability, power efficiency, and
% cost
% savings~\cite{blade-server,kona,the-machine,requirements,clio,firebox,leap,zombieland,storm,aifm,supernic}.
% By physically separating compute from memory across a network, it is possible to dynamically adjust
% hardware resource allocations to suit changing workloads.  A large number of
% proposals~\cite{infiniswap,fastswap,legoos,clover,sherman,farm,regions,fusee} have leveraged the
% remote direct memory access (RDMA) support found in modern network interface cards (NICs) due to its
% low latency, high throughput, and simple verb-based interface.  Yet, most share a common drawback:
% they do not support sharing remote memory across compute nodes.  For systems that
% do~\cite{clover,sherman,fusee}, even modest levels of write contention crater performance.

The reason why sharing remote memory is hard is easy to identify: sharing remote memory requires
coordinating access across multiple clients, yet the RDMA protocol---like TCP---provides a
connection-based abstraction; while connection-less operation is possible, much like UDP it provides
essentially no semantic guarantees.  Fundamentally, remote memory operations must be ordered to
provide coherent access, and the RDMA protocol provides two basic mechanisms to do so remotely
(i.e., in a 1-sided fashion): reliable connections that ensure ordering and atomic operations that
deliver mutual exclusion.  While the performance of RDMA connection handling has received
considerable attention~\cite{farm,storm,scalerpc}, connections remain an end-to-end abstraction, and
do not provide any guarantees regarding operations from distinct clients.  For that, systems must
rely on atomic operations like compare-and-swap (CAS), but their enhanced semantics dictate
expensive implementation choices on the NIC, dramatically restricting their performance compared to
simple verbs like read and write~\cite{design-guidelines}.  Moreover, atomic operations are
available only over reliable connections.

As a result, most existing systems that deliver scalable,
high-performance shared remote access depend on the presence of
computational resources collocated with the remote
data~\cite{herd,cell,farm,pilaf,storm}.  In particular, a
memory-local CPU can employ 2-sided RDMA operations to orchestrate
operations between multiple clients~\cite{herd,fasst}, avoiding the need
for atomic operations.  Unfortunately, such RPC-like approaches are
infeasible in the passive memory setting.  Alternatively,
organizations with significant resources have considered redesigning
the RDMA protocol itself to better support the needs of the
disaggregated usage case---by, e.g., removing the connection
abstraction and providing more powerful verbs~\cite{filemr,rma,star}---but such
hardware is not yet available.

In this chapter, we explore an alternative dimension: rather than relying
exclusively on end-to-end solutions, we consider leveraging in-network
resources---specifically programmable switches that are located between
clients and the remote memory servers---to accelerate systems based on
existing 1-sided RDMA verbs.  Concretely, we observe that in
rack-scale disaggregated settings, the top-of-rack (ToR) switch serves as a
single serialization point for all RDMA requests.  As a result, it is
possible to transparently rewrite RDMA operations in flight to
orchestrate requests from multiple clients to passive memory servers,
sidestepping the fundamental bottlenecks present in the current
connection-based RDMA protocol.


%% Resolving conflicts is hard because no centralized
%% serialization point exists. RDMA key-value stores use CPU's
%% on remote machines to serialize
%% writes~\cite{herd,cell,farm,pilaf,storm}, in the
%% disaggregated setting remote memory is not collocated with a
%% CPU, so no such serialization point exists. When clients are
%% collocated they can share an RDMA reliable connection which
%% provides in order delivery~\todo{Anil whats the RDMA
%% scheduling paper}, but in general, distributed clients
%% sharing remote memory must heavyweight RDMA atomic
%% operations to serialize requests, RDMA atomics are known to
%% not scale, and when issued by multiple clients to a shared
%% address, such as in the case of a lock, are 6x slower than
%% read/write verbs~\cite{design-guidelines}. However, at rack
%% scale, RDMA requests to remote memory are serialized, not
%% explicitly by a CPU, but implicitly when serialized on the
%% egress port of a TOR.

We present \sword, a top-of-rack switch that implements two separate yet complimentary approaches to
accelerating RDMA-based passive memory. Client-driven schemes must rely either on mutual exclusion
(i.e., locks) or optimistic concurrency control (which require multiple round trips to resolve
conflicts).  \sword\ removes the performance bottlenecks of both by 1) multiplexing multiple
clients' RDMA operations onto shared connections to leverage the ordering semantics delivered by
reliable connections~\cite{flock}, and 2) caching small amounts of metadata to dynamically steer
in-flight RDMA updates to serialize concurrent operations to remote-memory indexing structures.

We apply \sword\ to two remote memory systems that natively
support sharing: Sherman~\cite{sherman}, which uses locking, and Clover~\cite{clover} that relies on optimistic concurrency.
We show that both systems natively collapse under contention due to RDMA's
limitations, but \sword\ can remove their bottlenecks.  Concretely, 
%% We address these limitations with \sword. {\sword} resolves
%% conflicts to remote memory by interposing on, and modifying
%% in-flight RDMA requests. Small amounts of metadata are
%% cached (just enough to detect and resolve conflicts). \sword
%% requires no modifications to the existing systems to
%% provide a performance boost.  We demonstrate with Sherman
%% and Clover that both locking, and optimistic concurrency can
%% be accelerated by orders of magnitude. 
%% %%
by multiplexing all acquire and release operations for a shared lock
in Sherman onto a single reliable connection, \sword\ can
replace the client-issued compare-and-swap operations
with a lightweight writes, delivering a potential 10$\times$ throughput gain.  Performance gains are even higher in the case of Clover,
where
%% When access to a shared location (lock acquire, release) is
%% required we show that our middlebox can multiplex reliable
%% RDMA connections, similar to collocated requests, thus
%% removing the need for atomics entirely. Here we swap atomic
%% operations with writes for a potential 10x gain in
%% throughput. In optimistic schemes when requests are not
%% necessary destined for the same address we
we resolve update conflicts to Clover's internal, append-only metadata
index structure by steering requests to an advancing set of locations,
as if they had been issued by a single serialized client.
Our evaluation shows that
%{\sword} dramatically increases
%the performance of Clover in the presence of write
%contention: U
under a 50:50 read-write workload, throughput
rises by almost 35$\times$ while bandwidth usage and tail
latency drop by 16 and 300$\times$.


% There has been tremendous interest in resource disaggregation in
% recent years, with both academic and industrial researchers chasing
% the potential for increased scalability, power efficiency, and cost
% savings~\cite{blade-server,fastswap,rethinking,the-machine,requirements,clio-arxiv,firebox,leap,zombieland,storm,aifm,legoos,supernic}.
% By physically separating compute from storage across a network, it is
% possible to dynamically adjust hardware resource allocations to suit
% changing workloads.  Considerable headway has been made at higher
% levels of the storage hierarchy; published and even production systems
% support remoting spinning disks, SSDs, and modern non-volatile
% memory technologies~\cite{decible}.  Remote primary storage---a.k.a.
% memory pooling---remains a fundamental challenge, however, due to the
% orders-of-magnitude disparity between main-board access latency and
% even intra-rack round trips.

%Resource disaggregation is an architectural paradigm which separates
%disk, CPU and memory over a network. The goal of this architecture is
%to enable extreme flexibility in terms of machine composition.  For
%example a systems memory capacity can be dynamically apportioned by
%reconfiguration, rather than by manually changing the physical
%components of a single machine. It is now common for disks (HHD and
%SSD) to be disaggregated from CPU and memory. SSDs are comparatively
%easier to disaggregated than main memory as their access latencies are
%on the order of 10's of microseconds which amortizes the network round
%trip cost.

%Local memory latency is around 50ns. The cost of accessing memory over the
%network is on the order of 1us -- approximately a 20x overhead. This order of
%magnitude difference in latency makes hiding remote memory accesses a hard
%problem.  

% The hardware community has made great strides in closing the latency
% gap via novel technologies like silicon photonics and new rack-scale
% interconnects, but commercially available options remain significantly
% slower than on-board alternatives.  Concretely, while industrial consortia
% have proposed cache-coherent memory technologies~\cite{genz,cxl} that
% would dramatically lower access latencies, currently available
% interconnects based on RDMA~\cite{infiniband-spec}
%%
%\todo{distinguish CXL 200-500ns latencies have been proposed}
%%
% remain on the order of 20$\times$ slower than a local access (e.g.,
% 50~ns local versus 1~$\mu$s remote).  As a result, despite the fact
% that current-generation memory transport technologies provide the
% ability to directly execute requests like read, write, and compare-and
% swap-on remote host memory through the use of RDMA-capable
% NICs~\cite{connectx}, SoCs~\cite{cavium}, FPGA
% SmartNICs~\cite{corundum,kv-direct}, or DPUs~\cite{fungible}, most
% existing systems coordinate with a remote CPU on the socket at which
% the DMA is being performed to assist with
% serialization~\cite{cliquemap,erpc,herd,sonuma,storm}.
% %%


%% % and Omni-Path~\cite{omni-path}. // omni-path is dead now % 
%Each protocol, while distinct, meets approximately the same requirements,
%reliable access to byte addressable remote memory with low latency and high
%throughput.

%% it would be nice to Cite SUPERNIC and CLIO here but I'm not sure it makes
%sense untill it's published at a major venue
%%Clio~\cite{clio-arxiv}
%%todo ask alex about the archive reference
%%todo do a quick read of how DMA is dealt with on the other interconnects

% In the absence of a general-purpose CPU located alongside remote
% memory, it falls to each individual client to ensure that its reads
% and writes are serialized, usually by leveraging expensive
% hardware-provided atomic operations at the server like
% compare-and-swap (CAS)~\cite{design-guidelines} as the latencies
% involved in client-side coordination are prohibitive.  As a result,
% most existing systems simply partition memory completely and forgo
% sharing~\cite{reigons,fastswap,legoos}.
%%
%The few published systems that provide fully passive remote memory
%target scenarios involving read-heavy workloads~\cite{clover}, client
%colocation~\cite{sherman}, or memory-inefficient
%datastructures~\cite{race} \textbf{XXX:Need to say more} where the
%costs of conflict detection and resolution can be effectively
%amortized.
%
% The few published systems that do support shared access mediate
% requests to specialized data structures~\cite{clover,sherman}.

% For example, Sherman, a write-optimized
% B+Tree~\cite{sherman} places its locks in NIC memory at the server to
% avoid crossing the remote PCIe bus.
% %, resulting in 3$\times$ higher throughput.
% Clover~\cite{clover} implements a hash table that
% supports lock-less reads; concurrent writes are supported through a
% client-driven optimistic concurrency protocol.  Despite their clever
% designs, however, both approaches simply delay the inevitable:
% %%
% Sherman's NIC-based locks are subject to significant hardware limits
% imposed by the CAS operation required to enforce serialization.
% Similarly, Clover's client-based recovery scheme quickly becomes cost
% prohibitive when faced with non-trivial levels of write contention.
%
%it is repeatedly executed on a single
%address (Lock acquire and release). Further, Sherman requires that cliques of
%clients are colocated to resolve most contention based conflicts locally.
%%
% CAS instructions are only executed to commit
%writes and never land on the same address twice - effectively bypassing the
%single address limits of CAS.
%%
%While highly scalable for read-heavy workloads,  This

% We argue that these shortcomings are not unique to the particular
% systems, but rather fundamental to any approach that implements
% distributed conflict resolution.
%
%% minimize conflicts by caching  metadata about the
%% location of the latest writes and reads while also make use of a
%% remote data structures which allows for lock-less reads. In the case of
%% highly contended resources however the performance of clover
%% diminishes sharply due to an increased number of atomic locking
%% operations required on writes.
%
% The obvious alternative is to deploy a centralized memory controller
% (e.g., a CXL 2.0 switch) that can serve as a serialization point and
% ensure all races are resolved before accessing memory, but effective
% realization of such a design has proven elusive.  While many proposals
% exist, none of them have yet been implemented in commercially
% available hardware.  More to the point, such designs are inherently
% unscalable as they require all accesses to be managed by the
% controller, rather than forwarded directly between the client and
% relevant server.

% In this work we make the observation that such a serialization point
% already exists in today's rack-scale disaggregated deployments: the
% top-of-rack switch.  We propose to leverage the capabilities of modern
% programmable switches to cache sufficient information about in-flight
% requests to transparently detect and resolve conflicts before they
% occur.  Unlike a centralized memory controller, however,
% our serializer does not need to operate on---or even maintain state for---all
% remote memory requests.  First, it only needs to address actual conflicts
% and can avoid the unnecessary costs of enforcing ordering among
% unrelated requests.  Second, in deployment scenarios where it may
% lack the resources to track all requests, our serializer can serve as a
% performance-enhancing proxy: when deployed alongside client-based
% conflict resolution techniques, it can allow even conflicting requests
% to pass through unmodified without jeopardizing safety while
% decreasing the frequency of conflict resolution.

% We present {\sword}, an on-path serializer
%(implemented either
%directly on the top-of-rack switch or an attached
%middle box~\cite{disandapp})
% that dramatically improves the performance of remote memory systems
% that support write sharing.  Like all ToRs, {\sword} imposes a
% globally observable total order on memory requests (i.e., packets)
% within a rack.  In scenarios where {\sword} has the resources to
% explicitly manage RDMA connections, it can enforce per-server ordering
% at the ToR and remove the expensive CAS operations from all in-flight
% packets, avoiding their associated performance bottlenecks entirely.
% More generally, however, {\sword} can be deployed alongside an
% underlying optimistic concurrency scheme: remote memory operations
% remain guarded to ensure that clients can detect and recover from
% conflicts of which {\sword} may not be aware.  Instead, because
% {\sword} understands the disaggregated memory protocol, it can keep a
% cache of recent operations to adjust subsequent requests whose
% guards it knows are doomed to fail.  In such cases,
% %If suitably provisioned (i.e., it has the
% %appropriate metadata cached),
% it modifies requests in
% flight to account for the preceding operations and decrease the
% likelihood the guard will trip.

% We prototype {\sword} in two scenarios using a rack of servers
% equipped with ConnectX-5 RoCE-enabled NICs.  First, we use a
% DPDK-based implementation to replace CAS requests in flight with
% standard RDMA verbs, allowing systems like Sherman to overcome the
% hardware limit on atomic requests per queue pair.  Second, we
% implement a lightweight version of \sword\ on a P4 programmable switch
% to accelerate Clover's optimistic concurrency protocol. Our evaluation
% shows that {\sword} dramatically increases the performance of Clover
% in the presence of write contention: Under a 50:50 read-write
% workload, throughput rises by almost 35$\times$ while bandwidth usage and tail latency drop by 16 and 300$\times$.

% %% Using RDMA transport information and clover specific application
% %% knowledge all reads and writes to contended areas are totally ordered
% %% in the network. Specifically all reads and writes to the same keys are
% %% multiplexed to the same queue pairs, by utilizing the RDMA ordering
% %% requirements of QP's reads and writes require no expensive locks and
% %% can flow at line rate to remote memory. This ordering requires a
% %% number in band adjustments to the RDMA protocol in order to
% %% interoperable with commodity hardware. QP state must be maintained in
% %% network, specifically the sequence numbers of multiplexed requests, so
% %% that response packets can be demultiplexes back to their original
% %% connections. Small adjustments such as generating ACKs for collapsed
% %% requests is also required. We demonstrate that these algorithms are
% %% implementable in network at little cost with a DPDK prototype. We
% %% measure that ~\todo{we achieve a ?X improvement in performance using
% %%   only XMB of in network state, and ?X performance improvement in
% %%   highly contested settings with full use of system memory}.
