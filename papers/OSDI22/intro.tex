\section{Introduction}

There has been tremendous interest in resource disaggregation in
recent years, with both academic and industrial researchers chasing
the potential for increased scalability, power efficiency, and cost
savings~\cite{requirements,zombieland,legoos}.  By physically separating compute
from storage across a network, it is possible to dynamically adjust
hardware resource allocations to suit changing
workloads~\cite{fastswap}.  Considerable headway has been made at
higher levels of the storage hierarchy; published and even production
systems now support remoting spinning disks, SSDs, and modern
non-volatile memory technologies~\cite{decible}.  Remote primary
storage, however, remains a fundamental challenge due to the
orders-of-magnitude disparity between main-board memory accesses and
network round trips.

%Resource disaggregation is an architectural paradigm which separates
%disk, cpu and memory over a network. The goal of this architecture is
%to enable extreme flexibility in terms of machine composition.  For
%example a systems memory capacity can be dynamically apportioned by
%reconfiguration, rather than by manually changing the physical
%components of a single machine. It is now common for disks (HHD and
%SSD) to be disaggregated from CPU and memory. SSDs are comparatively
%easier to disaggregated than main memory as their access latencies are
%on the order of 10's of microseconds which amortizes the network round
%trip cost.

%Local memory latency is around 50ns. The cost of accessing memory over the
%network is on the order of 1us -- approximately a 20x overhead. This order of
%magnitude difference in latency makes hiding remote memory accesses a hard
%problem.  

The hardware community has made great strides in closing the latency
gap via novel technologies like silicon photonics and new rack-scale
interconnects, but commecially available options remain significanly
slower than on-board options.  Concretely, the latency of making a
memory access over commercially available interconnect technologies
like RDMA~\cite{infiniband-spec}, GenZ~\cite{genz}, and CXL~\cite{cxl}
remains on the order of 20$\times$ slower than a local access (i.e.,
50~ns local versus 1~$\mu$s remote).  As a result, despite the fact
that these memory transport technologies provide the ability to
directly execute operations like read, write and compare and swap on
remote host memory through the use of a RDMA capable
NICs~\cite{connectx}, SOC~\cite{cavium} or FPGA
smartNIC~\cite{corundum,kv-direct} or DPUs~\cite{fungible}, most
existing systems coordinate with the remote CPU on the socket at which
the DMA is being performed to assist with
serialization~\cite{cliquemap,erpc,herd,sonuma,storm}.  The very few
published systems that attempt to provide fully passive remote
memory~\cite{reigons,clover} focus on read-heavy workloads to
amoritize the high costs of conflict detection and resolution.

%% % and Omni-Path~\cite{omni-path}. // omni-path is dead now % 
%Each protocol, while distinct, meets approximately the same requirements,
%reliable access to byte addressable remote memory with low latency and high
%throughput.

%% it would be nice to Cite SUPERNIC and CLIO here but I'm not sure it makes
%sense untill it's published at a major venue
%%Clio~\cite{clio-arxiv}
%%todo ask alex about the archive reference
%%todo do a quick read of how dma is delt with on the other interconnects

In the absence of a general-purpose CPU located alongside remote
memory, it falls to each individual client to ensure that its reads
and writes are serialized, usually by leveraging expensive
hardware-provided atomoic operations like compare and
swap~\cite{design-guidelines,clover}.  The cost of client-side
serialization is stark, and most prior systems simply partition memory
completely and forgo sharing~\cite{reigons,fastswap, legoos}.  On the
other hand, one recent disaggregated key-value store,
Clover~\cite{clover}, goes to great lengths to provide lockless reads
while supporting concurrent writes by deploying an optimistic
concurrency scheme that leverages RDMA's atomic compare-and-swap
operation to detect and recover from write/write conflicts.  While
highly scalable for read-heavy workloads, Clover's client-based
recovery scheme quickly becomes cost prohibitive when faced with
significant levels of write contention.  This shortcoming is not
limited to Clover's design, however, but fundamental to any approach
based upon distributed conflict resolution.

%% minimize conflicts by caching  metadata about the
%% location of the latest writes and reads while also make use of a
%% remote data structures which allows for lockless reads. In the case of
%% highly contended resources however the performance of clover
%% diminishes sharply due to an increased number of atomic locking
%% operations required on writes.

The traditional alternative, of course, is to deploy a centralized
memory controller that can serve as a serializatio point and ensure
all races are resolved before accessing memory.  Unfortunately, such
designs are inherently unscalable as they require all accesses to be
routed through the controller, rather than forwarded directly between
the client and relevant server.  In this work we make the observation
that such a serialization point already exists in rack-scale
disaggregated deployments: the top-of-rack switch.  We propose to
leverage the capabilities of modern programmable switches to cache
sufficient information about in-flight operations to transparently
detect and resolve conflicts before they occur.  Unlike a traditonal
centralized memory controller, our serializer need only act upon
actual conflicts and can avoid the unnecessary costs of enforcing
ordering among unrelated operations.  Moreover, in our design the
serializer serves as a performance-enhancing proxy: it can always
allow even conflicting operations to pass through unmodified without
jeopardizing safey, as the default client-based conflict resolution
strategy remains.

We present {\sword}, an on-path serilaizer (implemented either
directly on the top-of-rack switch or an attached
middlebox~\cite{disandapp}) that dramatically improves the performance
of optimistic passive remote memory systems that support write
sharing.  Like all ToRs, {\sword} imposes a globally observable total
order on memory operations within a rack.  Crucially, {\sword} does
not replace the underlying optimistic concurrency scheme: all remote
memory operations are still suitably guarded to ensure that clients
can detect and recover from conflicts.  Rather, because {\sword}
understands the disaggregated memory protocol, it can inspect the
total ordering and detect which guards will fail.  If suitably
provisioned (i.e., it has the appropraite matadata cached), it can
transparently modify requests in flight to account for the preceeding
operations and decrease the liklihood the guard will trip.  Moreover,
if {\sword} is configured to explicitly manage the RDMA connections
themselves, it can further enforce per-server ordering and remove the
expensive guards entirely.

We prototype {\sword} in the context of Clover on ConnectX-5 RoCE RDMA
NICs remove the need for expensive atomic operations

on an RDMA enabled NIC. To show the benefits of such a system we
implement a variety of in network contention resolutions for the
Clover protocol and demonstrate their performance boots. Specially we
cache the locations of the most recent reads and writes with stale
metadata are steered to the most up to date locations allowing for
O(1) remote memory accesses in all cases.

We demonstrate in network serialization removes the need for expensive locking
operations on the NIC. Using RDMA transport information and clover specific
application knowledge all reads and writes to contended areas are totally
ordered in the network. Specifically all reads and writes to the same keys are
multiplexed to the same queue pairs, by utilizing the RDMA ordering requirements
of QP's reads and writes require no expensive locks and can flow at line rate to
remote memory. This ordering requires a number in band adjustments to the RDMA
protocol in order to interoperable with commodity hardware. QP state must be
maintained in network, specifically the sequence numbers of multiplexed
requests, so that response packets can be demultiplexes back to their original
connections. Small adjustments such as generating acks for collapsed requests is
also required. We demonstrate that these algorithms are implementable in network
at little cost with a DPDK prototype. We measure that ~\todo{we achieve a ?X
improvement in performance using only XMB of in network state, and ?X
performance improvement in highly contested settings with full use of system
memory}.
