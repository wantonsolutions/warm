\section{Limitations}

The main limitation of our work is that our middle box is developed in user
space software and not on programmable networking hardware. There are additional
limitations, in terms of language restrictions and processing power which occur
on real hardware but not in our DPDK prototype. For example in programmable
switches the need to recirculate packets which exceed the computational capacity
of a pipeline results in decreased overall bandwidth. We use prior work on
programable switches to inform our DPDK prototype and intentionally avoid
decisions which would not be implementable on a programmable switch.

Our experiments fall short of the underlying hardware limits. Our results are
relative and show real performance boots which are obtained by reducing hardware
contention. In future work we would like to extend our measurements to push the
limitations of the underlying hardware. We expect that measuring at line rate
will only increase the benifit seen by reducing this contention based on our
current measurements~\ref{fig:full_system_performance}.

Some aspects of RDMA interposition are left out of this work. For instance how
to deal with ECN packets correctly id a difficult question when connections are
being multiplexed as the generated ECN packet has a single destination. One
option is to broadcast the ECN to all clients multiplexed on the destination
connection and allow end-to-end congestion control. Another is to keep track of
individual client request rates and only issue ECN to the highest requesting
clients. We leave the finer points of congestion control to future work.