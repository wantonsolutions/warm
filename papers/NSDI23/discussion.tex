%\section{Discussion and future work}
%
While our prototype focuses on Clover, we believe the
%Our approach only measures the benefits which clover observers by using our in
%network contention resolution. However our
approach is general, and in the future we would like to add support
for additional disaggregated memory systems. For example Remote
Regions~\cite{reigons} implements a POSIX file system API for remote
memory.
%Our approach could also be used to remove contention in
%a much more standardized remote memory environment.

More broadly, each of these systems is designed with a different set
of assumptions than we anticipate are appropriate for future
deployments.  As others have also observed~\cite{mom}, in-network
serialization occurs naturally in today's network fabrics, so it seems
fruitful to consider how to design algorithms and data structures
under the assumption most conflicts can be resolved using small
amounts of high-speed compute in the network itself. We aim to explore
the design space of such algorithms and data structures for
disaggregation in the future.

%%
\subsection{CXL} 

CXL is a CPU-to-Device interconnect layered on PCIe 5.0 designed to provide a
coherent interface to devices, accelerators and memory~\cite{cxl-spec}.  .
Although at the time of writing CXL is not a commercially available technology,
research prototypes have have reported remote memory access latencies between
200-426ns~\cite{direct-cxl,microsoft-cxl-first-gen,facebook-cxl-tpp}.

CXL removes the DIMM slot constraint on per server memory capacity and
provides a way forward for high capacity memory pooling between many CPUs. The CXL protocols do
not provide opportunities for reducing or removing contention to shared memory.
CLX.cache for instance exposes MESI states to the host CPU, and levees
coordination to them. This interface suggests that CXL devices will suffer from
identical lock contention problems as traditional memory with inflated costs for
coordination. Given that PCIe root complexes lack programmability our techniques
will likely not be applicable to CXL. We leave resolving CXL memory contention
to future work.